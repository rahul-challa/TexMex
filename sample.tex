% Sample LaTeX Document for TexMex Extension
% This file demonstrates various LaTeX features supported by TexMex

\documentclass{article}
\usepackage{amsmath}    % For mathematical equations
\usepackage{graphicx}   % For including images
\usepackage{hyperref}   % For hyperlinks
\usepackage{listings}   % For code listings

\title{Sample LaTeX Document}
\author{TexMex Extension}
\date{\today}

\begin{document}

\maketitle

\section{Introduction}
This is a sample LaTeX document to demonstrate the TexMex extension's features.
The extension provides live preview of your LaTeX documents as you edit them.

\section{Mathematical Formulas}
Here are some examples of mathematical formulas:

\subsection{Inline Math}
The famous equation $E = mc^2$ was proposed by Einstein.

\subsection{Display Math}
Here's a more complex equation:
\begin{equation}
    \int_{a}^{b} f(x) \, dx = F(b) - F(a)
\end{equation}

\section{Lists}
\subsection{Unordered List}
\begin{itemize}
    \item First item
    \item Second item
    \item Third item with \textbf{bold text}
\end{itemize}

\subsection{Ordered List}
\begin{enumerate}
    \item Numbered item 1
    \item Numbered item 2
    \item Numbered item 3
\end{enumerate}

\section{Tables}
\begin{table}[h]
    \centering
    \begin{tabular}{|c|c|c|}
        \hline
        \textbf{Header 1} & \textbf{Header 2} & \textbf{Header 3} \\
        \hline
        Cell 1 & Cell 2 & Cell 3 \\
        Cell 4 & Cell 5 & Cell 6 \\
        \hline
    \end{tabular}
    \caption{Sample Table}
    \label{tab:sample}
\end{table}

\section{Code Listing}
\begin{lstlisting}[language=Python]
def hello_world():
    print("Hello, World!")
    return True
\end{lstlisting}

\section{Cross References}
As shown in Table~\ref{tab:sample}, we can use cross-references.

\section{Links}
Visit the \href{https://github.com/RahulChalla/texmex}{TexMex GitHub repository} for more information.

\end{document} 